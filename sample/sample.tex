\documentclass[10pt,fontset=adobe,UTF8,twoside]{ctexrep}

\usepackage{lnotes}

\newcommand{\reporttitle}{学习笔记}


\begin{document}

%%
\pagestyle{fancy}
\fancyhf{}
\fancyhead[RE]{\normalfont\small\hei\nouppercase{\leftmark}}
\fancyhead[LO]{\normalfont\small\kai\nouppercase{\rightmark}}
\fancyhead[LE,RO]{\thepage}

\makeatletter
\@openrightfalse
\makeatother

%\frontmatter
\mktitle{學習筆記}{斗為帝車}

\begin{cabstract}
   這是一篇學習《史記 天官書》的學習筆記,這是一篇學習《史記 天官書》的學習筆記,這是一篇學習《史記 天官書》的學習筆記,
   這是一篇學習《史記 天官書》的學習筆記,這是一篇學習《史記 天官書》的學習筆記,這是一篇學習《史記 天官書》的學習筆記。
\end{cabstract}

\tableofcontents
    
%\mainmatter

\chapter{史記}

\section{天官書}

中宮天極星,其一明者,太一常居也;旁三星三公,或曰子屬。後句四星,末大星正妃,餘三星,後宮之屬也。環之匡衛十二星,藩臣。皆曰紫宮。

前列直斗口三星,隨北端兌,若見若不,曰陰德,或曰天一。紫宮左三星曰天槍,右五星曰天棓,後六星絕漢抵營室,曰閣道。

北斗七星,所謂「旋、璣、玉衡以齊七政」。杓攜龍角,衡殷北斗*,魁枕參首。用昏建者杓;杓,自華以西南。夜半建者衡;衡,殷中州河、濟之間。平旦建者魁;魁,海岱以東北也。斗為帝車,運於中央,臨制四鄉。分陰陽,建四時,均五行,移節度,定諸紀,皆繫於斗。

斗魁戴匡六星,曰文昌宮:一曰上將,二曰次將,三曰貴相,四曰司命,五曰司中,六曰司祿。在斗魁中,貴人之牢。魁下六星,兩兩相比者,名曰三能。三能色齊,君臣和;不齊,為乖戾。輔星明近,輔臣親強;斥小,疏弱。

杓端有兩星:一內為矛,招搖;一外為盾,天鋒。有句圜十五星,屬杓,曰賤人之牢。其牢中星實則囚多,虛則開出。

天一、槍、棓、矛、盾動搖,角大,兵起。

東宮蒼龍,房、心。心為明堂,大星天王,前後星子屬。不欲直,直則天王失計。房為府,曰天駟。其陰,右驂。旁有兩星曰衿;北一星曰舝。東北曲十二星曰旗。旗中四星天市;中六星曰市樓。市中星眾者實;其虛則秏。房南眾星曰騎官。

左角,李;右角,將。大角者,天王帝廷。其兩旁各有三星,鼎足句之,曰攝提。攝提者,直斗杓所指,以建時節,故曰「攝提格」。亢為疏廟,主疾。其南北兩星*,曰南門。氐為天根,主疫。

尾為九子,曰君臣;斥絕,不和。箕為敖客,曰口舌。

火犯守角,則有戰。房、心,王者惡之也。

南宮朱鳥,權、衡。衡,太微,三光之廷。匡衛十二星,藩臣:西,將;東,相;南四星,執法;中,端門;門左右,掖門。門內六星,諸侯。其內五星,五帝坐。後聚一十五星,蔚然,曰郎位;傍一大星,將位也。月、五星順入,軌道,司其出,所守,天子所誅也。其逆入,若不軌道,以所犯命之;中坐,成形,皆群下從謀也。金、火尤甚。廷藩西有隋星五,曰少微,士大夫。權,軒轅。軒轅,黃龍體。前大星,女主象;旁小星,御者後宮屬。月、五星守犯者,如衡占。

東井為水事。其西曲星曰鉞。鉞北,北河;南,南河;兩河、天闕間為關梁。輿鬼,鬼祠事;中白者為質。火守南北河,兵起,穀不登。故德成衡,觀成潢,傷成鉞,禍成井,誅成質。

柳為鳥注,主木草。七星,頸,為員官。主急事。張,素,為廚,主觴客。翼為羽翮,主遠客。

軫為車,主風。其旁有一小星,曰長沙,星星不欲明;明與四星等,若五星入軫中,兵大起。軫南眾星曰天庫樓;庫有五車。車星角若益眾,及不具,無處車馬。

西宮咸池,曰天五橫。五潢,五帝車舍。火入,旱;金,兵;水,水。中有三柱;柱不具,兵起。

奎曰封豕,為溝瀆。婁為聚眾。胃為天倉。其南眾星曰廥積。

昴曰旌頭,胡星也,為白衣會。畢曰罕車,為邊兵,主弋獵。其大星旁小星為附耳。附耳搖動,有讒亂臣在側。昴、畢間為天街。其陰,陰國;陽,陽國。

參為白虎。三星直者,是為衡石。下有三星,兌,曰罰,為斬艾事。其外四星,左右肩股也。小三星隅置,曰觜觿,為虎首,主葆旅事。其南有四星,曰天廁。廁下一星,曰天矢。矢黃則吉;青、白、黑,凶。其西有句曲九星,三處羅:一曰天旗,二曰天苑,三曰九斿。其東有大星曰狼。狼角變色,多盜賊。下有四星曰狐,直狼。狼比地有大星,曰南極老人。老人見,治安;不見,兵起。常以秋分時候之於南郊。附耳入畢中,兵起。

北宮玄武,虛、危。危為蓋屋;虛為哭泣之事。

其南有眾星,曰羽林天軍。軍西為壘,或曰鉞。旁有一大星為北落。北落若微亡,軍星動角益希,及五星犯北落,入軍,軍起。火、金、水尤甚:火,軍憂;水,[水]患;木、土,軍吉。危東六星,兩兩相比,曰司空。

營室為清廟,曰離宮、閣道。漢中四星,曰天駟。旁一星,曰王良。王良策馬,車騎滿野。旁有八星,絕漢,曰天潢。天潢旁,江星。江星動,人涉水。

杵、臼四星,在危南。匏瓜,有青黑星守之,魚鹽貴。

南斗為廟,其北建星。建星者,旗也。牽牛為犧牲。其北河鼓。河鼓大星,上將;左右,左右將。婺女,其北織女。織女,天女孫也。

察日月之行,以揆歲星順逆。曰東方木,主春,曰甲乙*。義失者,罰出歲星。歲星贏縮,以其舍命國。所在國不可伐,可以罰人。其趨舍而前曰贏,退舍曰縮。贏,其國有兵不復;縮,其國有憂,將亡,國傾敗。其所在,五星皆從而聚於一舍,其下之國可以義致天下。

以攝提格歲:歲陰左行在寅,歲星右轉居丑。正月,與斗、牽牛晨出東方,名曰監德。色蒼蒼有光。其失次,有應見柳。歲早,水;晚,旱。

歲星出,東行十二度,百日而止,反逆行;逆行八度,百日,復東行。歲行三十度十六分度之七,率日行十二分度之一,十二歲而周天。出常東方,以晨;入於西方,用昏。

單閼歲:歲陰在卯,星居子。以二月與婺女、虛、危晨出,曰降入。大有光。其失次,有應見張。名曰降入,其歲大水。

執徐歲:歲陰在辰,星居亥。以三月居與營室、東壁晨出,曰青章。青青甚章。其失次;有應見軫。曰青章,歲早,旱;晚,水。

大荒駱歲:歲陰在巳,星居戌。以四月與奎、婁、胃、昴晨出,曰跰踵。熊熊赤色,有光。其失次,有應見亢。

敦牂歲:歲陰在午,星居酉。以五月與胃、昴、畢晨出,曰開明。炎炎有光。偃兵;唯利公王,不利治兵。其失次,有應見房。歲早,旱;晚,水。

叶洽歲:歲陰在未,星居申。以六月與觜觿、參晨出,曰長列。昭昭有光。利行兵。其失次,有應見箕。

涒灘歲:歲陰在申,星居未。以七月與東井、輿鬼晨出,曰大音。昭昭白。其失次,有應見牽牛。

作鄂歲:歲陰在酉,星居午。以八月與柳、七星、張晨出,曰為長王。作作有芒。國其昌,熟穀。其失次,有應見危。曰大章,有旱而昌,有女喪,民疾。

閹茂歲:歲陰在戌,星居巳。以九月與翼、軫晨出,曰天睢。白色大明。其失次,有應見東壁。歲水,女喪。

大淵獻歲:歲陰在亥,星居辰。以十月與角、亢晨出,曰大章。蒼蒼然,星若躍而陰出旦,是謂正平。起師旅,其率必武;其國有德,將有四海。其失次,有應見婁。

困敦歲:歲陰在子,星居卯。以十一月與氐、房、心晨出,曰天泉。玄色甚明。江池其昌,不利起兵。其失次,有應在昴。

赤奮若歲:歲陰在丑,星居寅,以十二月與尾、箕晨出,曰天皓。黫然黑色甚明。其失次,有應見參。

當居不居,居之又左右搖,未當去去之,與他星會,其國凶。所居久,國有德厚。其角動,乍小乍大,若色數變,人主有憂。

其失次舍以下,進而東北,三月生天棓,長四丈,末兌,進而東南,三月生彗星,長二丈,類彗星ppp15L3。退而西北,三月生天欃,長四丈,末兌。退而西南,三月生天槍,長數丈,兩頭兌。謹視其所見之國,不可舉事用兵。其出如浮如沈,其國有土功;如沈如浮,其野亡。色赤而有角,其所居國昌。迎角而戰者,不勝。星色赤黃而沈,所居野大穰。色青白而赤灰,所居野有憂。歲星入月,其野有逐相;與太白斗,其野有破軍。

歲星一曰攝提,曰重華,曰應星,曰紀星。營室為清廟,歲星廟也。

察剛氣以處熒惑。曰南方火,主夏,日丙、丁。禮失,罰出熒惑,熒惑失行是也。出則有兵,入則兵散。以其捨命國。(熒惑)熒惑為勃亂,殘賊、疾、喪、饑、兵。反道二舍以上,居之,三月有殃,五月受兵,七月半亡地,九月太半亡地。因與俱出入,國絕祀。居之,殃還至,雖大當小;久而至,當小反大。其南為丈夫[喪],北為女子喪。若角動繞環之,及乍前乍後,左右,殃益大。與他星斗,光相逮,為害;不相逮,不害。五星皆從而聚于一舍,其下國可以禮致天下。

法,出東行十六舍而止;逆行二舍;六旬,復東行,自所止數十舍,十月而入西方;伏行五月,出東方。其出西方曰「反明」,主命者惡之。東行急,一日行一度半。

其行東、西、南、北疾也。兵各聚其下;用戰,順之勝,逆之敗。熒惑從太白,軍憂;離之,軍卻。出太白陰,有分軍;行其陽,有偏將戰。當其行,太白逮之,破軍殺將。其入守犯太微、軒轅、營室,主命惡之。心為明堂,熒惑廟也。謹候此。

歷斗之會以定填星之位。曰中央土,主季夏,日戊、己,黃帝,主德,女主象也。歲填一宿,其所居國吉。未當居而居,若已去而復還,還居之,其國得土,不乃得女。若當居而不居,既已居之,又西東去,其國失土,不乃失女,不可舉事用兵。其居久,其國福厚;易,福薄。

其一名曰地侯,主歲。歲行十(二)[三]度百十二分度之五,日行二十八分度之一,二十八歲周天。其所居,五星皆從而聚于一舍,其下之國,可[以]重致天下。禮、德、義、殺、刑盡失,而填星乃為之動搖。

贏,為王不寧;其縮,有軍不復。填星,其色黃,九芒,音曰黃鐘宮。其失次上二三宿曰贏,有主命不成,不乃大水。失次下二三宿曰縮,有後戚,其歲不復,不乃天裂若地動。

斗為文太室,填星廟,天子之星也。

木星與土合,為內亂。饑,主勿用戰,敗;水則變謀而更事;火為旱;金為白衣會若水。金在南曰牝牡,年谷熟,金在北,歲偏無。火與水合為焠,與金合為鑠,為喪,皆不可舉事,用兵大敗。土為憂,主孽卿;大饑,戰敗,為北軍,軍困,舉事大敗。土與水合,穰而擁閼,有覆軍,其國不可舉事。出,亡地;入,得地。金為疾,為內兵,亡地。三星若合,其宿地國外內有兵與喪,改立公王。四星合,兵喪并起,君子憂,小人流。五星合,是為易行,有德,受慶,改立大人,掩有四方,子孫蕃昌;無德,受殃若亡。五星皆大,其事亦大;皆小,事亦小。

蚤出者為贏,贏者為客。晚出者為縮,縮者為主人。必有天應見於杓星。同舍為合。相陵為斗,七寸以內必之矣。

五星色白圜,為喪旱;赤圜,則中不平,為兵;青圜,為憂水;黑圜,為疾,多死;黃圜,則吉。赤角犯我城,黃角地之爭,白角哭泣之聲,青角有兵憂,黑角則水。意,行窮兵之所終。五星同色,天下偃兵,百姓寧昌。春風秋雨,冬寒夏暑,動搖常以此。

填星出百二十日而逆西行,西行百二十日反東行。見三百三十日而入,入三十日復出東方。太歲在甲寅,鎮星在東壁,故在營室。

察日行以處位太白。曰西方,秋,(司兵月行及天矢)日庚、辛,主殺。殺失者,罰出太白。太白失行,以其捨命國。其出行十八舍二百四十日而入。入東方,伏行十一舍百三十日;其入西方,伏行三舍十六日而出。當出不出,當入不入,是謂失舍,不有破軍,必有國君之篡。

其紀上元,以攝提格之歲,與營室晨出東方,至角而入;與營室夕出西方,至角而入;與角晨出,入畢;與角夕出,入畢;與畢晨出,入箕;與畢夕出,入箕;與箕晨出,入柳;與箕夕出,入柳;與柳晨出,入營室;與柳夕出,入營室。凡出入東西各五,為八歲,二百二十日,復與營室晨出東方。其大率,歲一周天。其始出東方,行遲,率日半度,一百二十日,必逆行一二舍;上極而反,東行,行日一度半,一百二十日入。其庳,近日,曰明星,柔;高,遠日,曰大囂,剛。其始出西[方],行疾,率日一度半,百二十日;上極而行遲,日半度,百二十日,旦入,必逆行一二舍而入。其庳,近日,曰大白,柔;高,遠日,曰大相,剛。出以辰、戌,入以丑、未。

當出不出,未當入而入,天下偃兵,兵在外,入。未當出而出,當入而不入,[天]下起兵,有破國。其當期出也,其國昌。其出東為東,入東為北方;出西為西,入西為南方。所居久,其鄉利;(疾)[易],其鄉凶。

出西(逆行)至東,正西國吉。出東至西,正東國吉。其出不經天;經天,天下革政。

小以角動,兵起。始出大,後小,兵弱;出小,後大,兵彊。出高,用兵深吉,淺凶;庳,淺吉,深凶。日方南金居其南,日方北金居其北,曰贏,侯王不寧,用兵進吉退凶。日方南金居其北,日方北金居其南,曰縮,侯王有憂,用兵退吉進凶。用兵象太白:太白行疾,疾行;遲,遲行。角,敢戰。動搖躁,躁。圜以靜,靜。順角所指,吉;反之,皆凶。出則出兵,入則入兵。赤角,有戰;白角,有喪;黑圜角,憂,有水事;青圜小角,憂,有木事;黃圜和角,有土事,有年。其已出三日而復,有微入,入三日乃復盛出,是謂耎,其下國有軍敗將北。其已入三日又復微出,出三日而復盛入,其下國有憂;師有糧食兵革,遺人用之;卒雖眾,將為人虜。其出西失行,外國敗;其出東失行,中國敗。其色大圜黃滜,可為好事;其圜大赤,兵盛不戰。

太白白,比狼;赤,比心;黃,比參左肩;蒼,比參右肩;黑,比奎大星。五星皆從太白而聚乎一舍,其下之國可以兵從天下。居實,有得也;居虛,無得也。行勝色,色勝位,有位勝無位,有色勝無色,行得盡勝之。出而留桑榆閒,疾其下國。上而疾,未盡其日,過參天,疾其對國。上復下,下復上,有反將。其入月,將僇。金、木星合,光,其下戰不合,兵雖起而不鬬;合相毀,野有破軍。出西方,昏而出陰,陰兵彊;暮食出,小弱;夜半出,中弱;雞鳴出,大弱:是謂陰陷於陽。其在東方,乘明而出陽,陽兵之彊,雞鳴出,小弱;夜半出,中弱;昏出,大弱:是謂陽陷於陰。太白伏也,以出兵,兵有殃。其出卯南,南勝北方;出卯北,北勝南方;正在卯,東國利。出酉北,北勝南方;出酉南,南勝北方;正在酉,西國勝。

其與列星相犯,小戰;五星,大戰。其相犯,太白出其南,南國敗;出其北,北國敗。行疾,武;不行,文。色白五芒,出蚤為月蝕,晚為天夭及彗星,將發其國。出東為德,舉事左之迎之,吉。出西為刑,舉事右之背之,吉。反之皆凶。太白光見景,戰勝。晝見而經天,是謂爭明,彊國弱,小國彊,女主昌。

亢為疏廟,太白廟也。太白,大臣也,其號上公。其他名殷星、太正、營星、觀星、宮星、明星、大衰、大澤、終星、大相、天浩、序星、月緯。大司馬位謹候此。

察日辰之會,以治辰星之位。曰北方水,太陰之精,主冬,日壬、癸。刑失者,罰出辰星,以其宿命國。

是正四時:仲春春分,夕出郊奎、婁、胃東五舍,為齊;仲夏夏至,夕出郊東井、輿鬼、柳東七舍,為楚;仲秋秋分,夕出郊角、亢、氐、房東四舍,為漢;仲冬冬至,晨出郊東方,與尾、箕、斗、牽牛俱西,為中國。其出入常以辰、戌、丑、未。

其蚤,為月蝕;晚,為彗星及天夭。其時宜效不效為失,追兵在外不戰。一時不出,其時不和;四時不出,天下大饑。其當效而出也,色白為旱,黃為五穀熟,赤為兵,黑為水。出東方,大而白,有兵於外,解。常在東方,其赤,中國勝;其西而赤,外國利。無兵於外而赤,兵起。其與太白俱出東方,皆赤而角,外國大敗,中國勝;其與太白俱出西方,皆赤而角,外國利。五星分天之中,積于東方,中國利;積于西方,外國用[兵]者利。五星皆從辰星而聚于一舍,其所捨之國可以法致天下。辰星不出,太白為客;其出,太白為主。出而與太白不相從,野雖有軍,不戰。出東方,太白出西方;若出西方,太白出東方,為格,野雖有兵不戰。失其時而出,為當寒反溫,當溫反寒。當出不出,是謂擊卒,兵大起。其入太白中而上出,破軍殺將,客軍勝;下出,客亡地。辰星來抵太白,太白不去,將死。正旗上出,破軍殺將,客勝;下出,客亡地。視旗所指,以命破軍。其繞環太白,若與鬬,大戰,客勝。兔過太白,閒可椷劍,小戰,客勝。兔居太白前,軍罷;出太白左,小戰;摩太白,有數萬人戰,主人吏死;出太白右,去三尺,軍急約戰。青角,兵憂;黑角,水。赤行窮兵之所終。

兔七命,曰小正、辰星、天欃、安周星、細爽、能星、鉤星。其色黃而小,出而易處,天下之文變而不善矣。兔五色,青圜憂,白圜喪,赤圜中不平,黑圜吉。赤角犯我城,黃角地之爭,白角號泣之聲。

其出東方,行四舍四十八日,其數二十日,而反入于東方;其出西方,行四舍四十八日,其數二十日,而反入于西方。其一候之營室、角、畢、箕、柳。出房、心閒,地動。

辰星之色:春,青黃;夏,赤白;秋,青白,而歲熟;冬,黃而不明。即變其色,其時不昌。春不見,大風,秋則不實。夏不見,有六十日之旱,月蝕。秋不見,有兵,春則不生。冬不見,陰雨六十日,有流邑,夏則不長。

角、亢、氐,兗州。房、心,豫州。尾、箕,幽州。斗,江、湖。牽牛、婺女,楊州。虛、危,青州。營室至東壁,并州。奎、婁、胃,徐州。昴、畢,冀州。觜觿、參,益州。東井、輿鬼,雍州。柳、七星、張,三河。翼、軫,荊州。

七星為員官,辰星廟,蠻夷星也。

兩軍相當,日暈;暈等,力鈞;厚長大,有勝;薄短小,無勝。重抱大破無。抱為和,背[為]不和,為分離相去。直為自立,立侯王;(指暈)[破軍](若曰)殺將。負且戴,有喜。圍在中,中勝;在外,外勝。青外赤中,以和相去;赤外青中,以惡相去。氣暈先至而後去,居軍勝。先至先去,前利後病;後至後去,前病後利;後至先去,前後皆病,居軍不勝。見而去,其發疾,雖勝無功。見半日以上,功大。白虹屈短,上下兌,有者下大流血。日暈制勝,近期三十日,遠期六十日。

其食,食所不利;復生,生所利;而食益盡,為主位。以其直及日所宿,加以日時,用命其國也。

月行中道,安寧和平。陰閒,多水,陰事。外北三尺,陰星。北三尺,太陰,大水,兵。陽閒,驕恣。陽星,多暴獄。太陽,大旱喪也。角、天門,十月為四月,十一月為五月,十二月為六月,水發,近三尺,遠五尺。犯四輔,輔臣誅。行南北河,以陰陽言,旱水兵喪。

月蝕歲星,其宿地,饑若亡。熒惑也亂,填星也下犯上,太白也彊國以戰敗,辰星也女亂。(食)[蝕]大角,主命者惡之;心,則為內賊亂也;列星,其宿地憂。

月食始日,五月者六,六月者五,五月復六,六月者一,而五月者五,凡百一十三月而復始。故月蝕,常也;日蝕,為不臧也。甲、乙,四海之外,日月不占。丙、丁,江、淮、海岱也。戊、己,中州、河、濟也。庚、辛,華山以西。壬、癸,恒山以北。日蝕,國君;月蝕,將相當之。

國皇星,大而赤,狀類南極。所出,其下起兵,兵彊;其沖不利。

昭明星,大而白,無角,乍上乍下。所出國,起兵,多變。

五殘星,出正東東方之野。其星狀類辰星,去地可六丈。

大賊星,出正南南方之野。星去地可六丈,大而赤,數動,有光。

司危星,出正西西方之野。星去地可六丈,大而白,類太白。

獄漢星,出正北北方之野。星去地可六丈,大而赤,數動,察之中青。此四野星所出,出非其方,其下有兵,沖不利。

四填星,所出四隅,去地可四丈。

地維咸光,亦出四隅,去地可三丈,若月始出。所見,下有亂;亂者亡,有德者昌。

燭星,狀如太白,其出也不行。見則滅。所燭者,城邑亂。

如星非星,如雲非雲,命曰歸邪。歸邪出,必有歸國者。

星者,金之散氣,[其]本曰火。星眾,國吉;少則凶。

漢者,亦金之散氣,其本曰水。漢,星多,多水,少則旱,其大經也。

天鼓,有音如雷非雷,音在地而下及地。其所往者,兵發其下。

天狗,狀如大奔星,有聲,其下止地,類狗。所墮及,望之如火光炎炎沖天。其下圜如數頃田處,上兌者則有黃色,千里破軍殺將。

格澤星者,如炎火之狀。黃白,起地而上。下大,上兌。其見也,不種而穫;不有土功,必有大害。

蚩尤之旗,類彗而後曲,象旗。見則王者征伐四方。

旬始,出於北斗旁,狀如雄雞。其怒,青黑,象伏鱉。

枉矢,類大流星,蛇行而倉黑,望之如有毛羽然。

長庚,如一匹布著天。此星見,兵起。

星墜至地,則石也。河、濟之閒,時有墜星。

天精而見景星。景星者,德星也。其狀無常,常出於有道之國。

凡望雲氣,仰而望之,三四百里;平望,在桑榆上,千餘[里]二千里;登高而望之,下屬地者三千里。雲氣有獸居上者,勝。

自華以南,氣下黑上赤。嵩高、三河之郊,氣正赤。恒山之北,氣下黑下青。勃、碣、海、岱之閒,氣皆黑。江、淮之閒,氣皆白。

徒氣白。土功氣黃。車氣乍高乍下,往往而聚。騎氣卑而布。卒氣摶。前卑而後高者,疾;前方而後高者,兌;後兌而卑者,卻。其氣平者其行徐。前高而後卑者,不止而反。氣相遇者,卑勝高,兌勝方。氣來卑而循車通者,不過三四日,去之五六里見。氣來高七八尺者,不過五六日,去之十餘里見。氣來高丈餘二丈者,不過三四十日,去之五六十里見。

稍云精白者,其將悍,其士怯。其大根而前絕遠者,當戰。青白,其前低者,戰勝;其前赤而仰者,戰不勝。陣雲如立垣。杼雲類杼。軸雲摶兩端兌。杓雲如繩者,居前亙天,其半半天。其蛪者類闕旗故。鉤雲句曲。諸此雲見,以五色合占。而澤摶密,其見動人,乃有占;兵必起,合斗其直。

王朔所候,決於日旁。日旁雲氣,人主象。皆如其形以占。

故北夷之氣如群畜穹閭,南夷之氣類舟船幡旗。大水處,敗軍場,破國之虛,下有積錢,金寶之上,皆有氣,不可不察。海旁蜄氣象樓臺;廣野氣成宮闕然。雲氣各象其山川人民所聚積。

故候息秏者,入國邑,視封疆田疇之正治,城郭室屋門戶之潤澤,次至車服畜產精華。實息者,吉;虛秏者,凶。

若煙非煙,若雲非雲,郁郁紛紛,蕭索輪囷,是謂卿雲。卿雲[見],喜氣也。若霧非霧,衣冠而不濡,見則其域被甲而趨。

(天)[夫]雷電、蝦虹、辟歷、夜明者,陽氣之動者也,春夏則發,秋冬則藏,故候者無不司之。

天開縣物,地動坼絕。山崩及徙,川塞谿垘;水澹(澤竭)地長,[澤竭]見象。城郭門閭,閨臬[枯槁]槁枯;宮廟邸第,人民所次。謠俗車服,觀民飲食。五穀草木,觀其所屬。倉府廄庫,四通之路。六畜禽獸,所產去就;魚鱉鳥鼠,觀其所處。鬼哭若呼,其人逢俉。化言,誠然。

凡候歲美惡,謹候歲始。歲始或冬至日,產氣始萌。臘明日,人眾卒歲,一會飲食,發陽氣,故曰初歲。正月旦,王者歲首;立春日,四時之(卒)始也。四始者,候之日。

而漢魏鮮集臘明正月旦決八風。風從南方來,大旱;西南,小旱;西方,有兵;西北,戎菽為,小雨,趣兵;北方,為中歲;東北,為上歲;東方,大水;東南,民有疾疫,歲惡。故八風各與其沖對,課多者為勝。多勝少,久勝亟,疾勝徐。旦至食,為麥;食至日昳,為稷;昳至餔,為黍;餔至下餔,為菽;下餔至日入,為麻。欲終日(有雨)有雲,有風,有日。日當其時者,深而多實;無雲有風日,當其時,淺而多實;有雲風,無日,當其時,深而少實;有日,無雲,不風,當其時者稼有敗。如食頃,小敗;熟五斗米頃,大敗。則風復起,有雲,其稼復起。各以其時用雲色占種(其)所宜。其雨雪若寒,歲惡。

是日光明,聽都邑人民之聲。聲宮,則歲善,吉;商,則有兵;徵,旱;羽,水;角,歲惡。

或從正月旦比數雨。率日食一升,至七升而極;過之,不占。數至十二日,日直其月,占水旱。為其環(城)[域]千里內占,則(其)為天下候,竟正月。月所離列宿,日、風、雲,占其國。然必察太歲所在。在金,穰;水,毀;木,饑;火,旱。此其大經也。

正月上甲,風從東方,宜蠶;風從西方,若旦黃雲,惡。

冬至短極,縣土炭,炭動,鹿解角,蘭根出,泉水躍,略以知日至,要決晷景。歲星所在,五穀逢昌。其對為沖,歲乃有殃。

太史公曰:自初生民以來,世主曷嘗不歷日月星辰?及至五家、三代,紹而明之,內冠帶,外夷狄,分中國為十有二州,仰則觀象於天,俯則法類於地。天則有日月,地則有陰陽。天有五星,地有五行。天則有列宿,地則有州域。三光者,陰陽之精,氣本在地,而圣人統理之。

幽厲以往,尚矣。所見天變,皆國殊窟穴,家占物怪,以合時應,其文圖籍禨祥不法。是以孔子論六經,紀異而說不書。至天道命,不傳;傳其人,不待告;告非其人,雖言不著。

昔之傳天數者:高辛之前,重、黎;於唐、虞,羲、和;有夏,昆吾;殷商,巫咸;周室,史佚、萇弘;於宋,子韋;鄭則裨灶;在齊,甘公;楚,唐眛;趙,尹皋;魏,石申。

夫天運,三十歲一小變,百年中變,五百載大變;三大變一紀,三紀而大備:此其大數也。為國者必貴三五。上下各千歲,然後天人之際續備。

太史公推古天變,未有可考于今者。蓋略以春秋二百四十二年之閒,日蝕三十六,彗星三見,宋襄公時星隕如雨。天子微,諸侯力政,五伯代興,更為主命,自是之後,眾暴寡,大并小。秦、楚、吳、越,夷狄也,為彊伯。田氏篡齊,三家分晉,并為戰國。爭於攻取,兵革更起,城邑數屠,因以饑饉疾疫焦苦,臣主共憂患,其察禨祥候星氣尤急。近世十二諸侯七國相王,言從衡者繼踵,而皋、唐、甘、石因時務論其書傳,故其占驗淩雜米鹽。

二十八舍主十二州,斗秉兼之,所從來久矣。秦之疆也,候在太白,占於狼、弧。吳、楚之疆,候在熒惑,占於鳥衡。燕、齊之疆,候在辰星,占於虛、危。宋、鄭之疆,候在歲星,占於房、心。晉之疆,亦候在辰星,占於參罰。

及秦并吞三晉、燕、代,自河山以南者中國。中國於四海內則在東南,為陽;陽則日、歲星、熒惑、填星;占於街南,畢主之。其西北則胡、貉、月氏諸衣旃裘引弓之民,為陰;陰則月、太白、辰星;占於街北,昴主之。故中國山川東北流,其維,首在隴、蜀,尾沒于勃、碣。是以秦、晉好用兵,復占太白,太白主中國;而胡、貉數侵掠,獨占辰星,辰星出入躁疾,常主夷狄:其大經也。此更為客主人。熒惑為孛,外則理兵,內則理政。故曰「雖有明天子,必視熒惑所在」。諸侯更彊,時菑異記,無可錄者。

秦始皇之時,十五年彗星四見,久者八十日,長或竟天。其後秦遂以兵滅六王,并中國,外攘四夷,死人如亂麻,因以張楚并起,三十年之閒兵相駘藉,不可勝數。自蚩尤以來,未嘗若斯也。

項羽救鉅鹿,枉矢西流,山東遂合從諸侯,西坑秦人,誅屠咸陽。

漢之興,五星聚于東井。平城之圍,月暈參、畢七重。諸呂作亂,日蝕,晝晦。吳楚七國叛逆,彗星數丈,天狗過梁野;及兵起,遂伏尸流血其下。元光、元狩,蚩尤之旗再見,長則半天。其後京師師四出,誅夷狄者數十年,而伐胡尤甚。越之亡,熒惑守鬬;朝鮮之拔,星茀于河戍;兵征大宛,星茀招搖:此其犖犖大者。若至委曲小變,不可勝道。由是觀之,未有不先形見而應隨之者也。

夫自漢之為天數者,星則唐都,氣則王朔,占歲則魏鮮。故甘、石歷五星法,唯獨熒惑有反逆行;逆行所守,及他星逆行,日月薄蝕,皆以為占。

余觀史記,考行事,百年之中,五星無出而不反逆行,反逆行,嘗盛大而變色;日月薄蝕,行南北有時:此其大度也。故紫宮、房心、權衡、咸池、虛危列宿部星,此天之五官坐位也,為經,不移徙,大小有差,闊狹有常。水、火、金、木、填星,此五星者,天之五佐,為[經]緯,見伏有時,所過行贏縮有度。

日變修德,月變省刑,星變結和。凡天變,過度乃占。國君彊大,有德者昌;羽小,飾詐者亡。太上修德,其次修政,其次修救,其次修禳,正下無之。夫常星之變希見,而三光之占亟用。日月暈適,雲風,此天之客氣,其發見亦有大運。然其與政事俯仰,最近(大)[天]人之符。此五者,天之感動。為天數者,必通三五。終始古今,深觀時變,察其精粗,則天官備矣。

蒼帝行德,天門為之開。赤帝行德,天牢為之空。黃帝行德,天夭為之起。風從西北來,必以庚、辛。一秋中,五至,大赦;三至,小赦。白帝行德,以正月二十日、二十一日,月暈圍,常大赦載,謂有太陽也。一曰:白帝行德,畢、昴為之圍。圍三暮,德乃成;不三暮,及圍不合,德不成。二曰:以辰圍,不出其旬。黑帝行德,天關為之動。天行德,天子更立年;不德,風雨破石。三能、三衡者,天廷也。客星出天廷,有奇令。

\chapter{顏色表}

{\color{DarkBlue} DarkBlue}

{\color{darkblue} darkblue}

{\color{Blue2} Blue2}

{\color{Royal} Royal}

{\color{DarkGreen} DarkGreen}

{\color{Purple} Purple}

{\color{Gray} Gray}

{\color{wcolornotas} wcolornotas}

{\color{miverde} miverde}


{\color{codebg} codebg}


{\color{Cyan} Cyan}


{\color{red} red}

{\color{DarkRed} DarkRed}

{\color{DeepSkyBlue} DeepSkyBlue}

{\color{SkyBlue} SkyBlue}

{\color{Tomato} Tomato}

{\color{OrangeRed} OrangeRed}

{\color{Maroon} Maroon}

{\color{DarkSlateGray} DarkSlateGray}

{\color{AliceBlue} AliceBlue}

{\color{Snow} Snow}

{\color{black} black}


\chapter{實例}
\section{表格}
\begin{center}
\tabcaption{Table name}\label{table:tablename}
\vspace{1ex}
\begin{tabularx}{0.8\textwidth}{|X|X|X|}
    \hline
    \hei 字段 & \hei 格式 & \hei 说明 \\ \hline
    \code{ID} & \code{INTEGER} & 标志符 \\ \hline
    \code{Name} & \code{VARCHAR(20)} & 名称 \\ 
    \hline
\end{tabularx}
\end{center}


%%
%% 字段前后可以以 > 和 < 来添加格式
\begin{center}
\tabcaption{Table name}\label{table:tablename}
\vspace{1ex}
\begin{tabular}{|l|>{$}c<{$}|l|}
    \hline
    \hei 字段 & \hei 格式 & \hei 说明 \\ \hline
    平方 & x^2 & 平方 \\ \hline
    开方 & \sqrt{x} & 开方 \\ 
    \hline
\end{tabular}
\end{center}

\section{列表}
%% Unordered list
\begin{ul}
  \item item1
  \item item2
  \item item3
  \item item4   
\end{ul}

%% Ordered List
\begin{ol}
  \item item1
  \item item2
  \item item3
  \item item4   
\end{ol}

\end{document}